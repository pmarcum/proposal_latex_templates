\documentclass[usenames,dvipsnames,modern]{CLASS_FILES/aastex631}  %<=== might need to be altered, depending on where you put this file in your folders
\usepackage[top=0.9in, left=1in, right=1in, bottom=1.1in]{geometry}
\usepackage{amsmath}
\usepackage{multirow}
\usepackage{breakurl}
\usepackage{hyperref}
\usepackage{textcomp}
\usepackage{cleveref}[2012/02/15]
\usepackage{fancyhdr}
\usepackage{graphicx}
\usepackage{amssymb}
\usepackage[toc,page]{appendix}
\usepackage{floatrow}
% the below package allows one to easily adjust spacing within an itemized list
\usepackage{enumitem}

\usepackage{nicefrac}
\usepackage{CLASS_FILES/euclid}
% Table float box with bottom caption, box width adjusted to content
\newfloatcommand{capbtabbox}{table}[][\FBwidth]
\usepackage{blindtext}
\usepackage{capt-of}% or \usepackage{caption}
\usepackage{booktabs}
\usepackage{varwidth}
\usepackage{scrextend}
\usepackage{xcolor}
\usepackage{titlesec}
\usepackage{colortbl}
\usepackage{tablefootnote}
\usepackage{array}
\usepackage{tocloft}

% allow for wrapping long text strings inside a cell within a table
\newcolumntype{M}[1]{>{\begin{varwidth}[t]{#1}}l<{\end{varwidth}}}

% place a line at the top with the title to the right and the program name to the left
\definecolor{WatermarkColor}{HTML}{8887AD}
\pagestyle{fancy}
\fancyhf{}
\fancyhead[L]{\color{WatermarkColor}NASA APRA 2024}  % <====  NEEDS TO BE CHANGED TO WHATEVER YOU WANT TO APPEAR IN THE UPPER LEFT CORNER MARGIN AREA
\fancyhead[R]{\color{WatermarkColor}My Shortened Title} %<=== CHANGE TO WHATEVER YOU WANT TO APPEAR IN THE UPPER RIGHT CORNER MARGIN AREA (should be ~9 words or less, in order to fit on one line)
\fancyfoot[C]{\thepage}

\titleformat{\section}{\centering\color{NavyBlue}\normalfont\large\bfseries}{\thesection}{1em}{}
\titleformat{\subsection}{\centering\color{NavyBlue}\normalfont\large\itshape}{\thesubsection}{1em}{}
\titleformat{\subsubsection}{\centering\color{NavyBlue}\normalfont\normalsize}{\thesubsubsection}{1em}{}

\interfootnotelinepenalty=10000
\crefformat{footnote}{#2\footnotemark[#1]#3}
% PUT ANY DEFINITIONS YOU WANT TO USE IN THE TEXT, BELOW.  IF NONE OF BELOW IS APPLICABLE TO YOUR PROPOSAL YOU CAN DELETE BELOW LINES
\newcommand{\HI}{\ion{H}{1}}
\newcommand{\et}{et al.}
\newcommand{\kms}{km~s$^{-1}$}
\newcommand{\s}{$\sim$}
\newcommand{\n}{$-$}
\newcommand{\RST}{\emph{Roman}}
\newcommand{\Euclid}{\emph{Euclid}}
\newcommand{\Gaia}{\emph{Gaia}}
\newcommand{\Chandra}{\emph{Chandra}}
\newcommand{\Hubble}{HST}
\newcommand{\Spitzer}{\emph{Spitzer}}
\newcommand{\Galex}{\emph{GALEX}}
\newcommand{\MASS}{2MASS}
\newcommand{\WISE}{\emph{WISE}}
\newcommand{\SDSS}{SDSS}
\newcommand{\CSNG}{\emph{CSNG}}
\newcommand{\ud}{\,\mathrm{d}}
\newcommand{\ue}{\,\mathrm{e}}
\newcommand{\mpch}{\,{\it h}^{-1}\, {\rm Mpc}}
\newcommand{\smallIndent}{\textcolor{white}{$--$}}
% make the vertical space between paragraphs smaller than default
% THIS SPACING CAN BE ALTERED BY YOU, DEPENDING ON YOUR PREFERENCE (AND NEED FOR SPACE IN THE TEXT!)
\addtolength{\parskip}{-0pt}
\newcommand{\magarc}{mag arcsec\ensuremath{^{\mathrm{-2}}}}
% generate a section that increments the section number counter but does not put a number by the section label nor lists the "fake" section in the table of contents.  
\newcommand{\nosection}[1]{%
  \refstepcounter{section}%
  \markright{#1}}

% THE BELOW IS OPTIONAL, ALLOWS A TEAM MEMBER TO PUT A COLORED COMMENT INTO THE LATEX ITSELF
\def\square{\vrule height 4.5pt width 4pt depth -0.5pt}
\def\threesquares{\square~\square~\square\ }
\def\remark#1{{\threesquares\tt (nota) #1~\threesquares}}

\newcommand{\todo}[1]{{\color{red} [{TODO: #1}]}}

% suppress new paragraph indentation
% THE BELOW IS OPTIONAL.  IF YOU WANT EACH PARAGRAPH TO BE INDENTED, COMMENT OUT BELOW
\setlength\parindent{0pt}

% put any extra white space appearing on a page at the bottom of the page
\raggedbottom

% make the references appear as "blah blah blah [1,3] instead of (1;3)
%\setcitestyle{square}
%\setcitestyle{citesep={,}}

% THE BELOW ARE NICE TO USE IN A PROPOSAL.  IF YOU HAVE A SHORT SENTENCE OR TWO THAT YOU WANT TO EMPHASIZE AS A CORE TAKE AWAY
% POINT, THEN PUT THEM INTO A COLORED BOX
% create a callout box
\usepackage{wrapfig}
\usepackage{tcolorbox}
\newtcolorbox{callout}[4][]
{
colback = #3!5!white,
colframe = #3!75!black,
fonttitle = \bfseries,
title = {#4},
#2,
}
\setlength\cftparskip{3pt}
\setlength\cftbeforesecskip{2pt}
\setlength\cftaftertoctitleskip{2pt}

% ================================  BEGINNING OF ACTUAL PROPOSAL =================================
\begin{document}
% science justification goes in here --  typically a 15-page limit, BUT CHECK SOLICITATION TO BE SURE!

\tableofcontents
\addtocontents{toc}{~\hfill\textbf{Page}\par}
\pagenumbering{roman}
%\renewcommand{\baselinestretch}{1.0}\normalsize
\newpage

%%%%%%%%%%%%%%%%%%%%%%%%%%%%%%%%%%%%%%%%%%%%%%%%%%%%%%%%%%%%%%%%%%%%%%%%
%%%%    S C I E N C E / T E C H N I C A L / M A N A G E M E N T    %%%%%
%%%%%%%%%%%%%%%%%%%%%%%%%%%%%%%%%%%%%%%%%%%%%%%%%%%%%%%%%%%%%%%%%%%%%%%%
%            >>>>>>>>>>  COMPONENT #2 <<<<<<<<<<<<<<<<<
\pagenumbering{arabic}
\addcontentsline{toc}{section}{\Large Scientific/Technical/Management Plan}
\begin{center}\color{NavyBlue}\normalfont\Large\textsc{Scientific/Technical/Management Plan}\end{center}
\vspace*{-0.5cm}

% ================================
% SUMMARY THAT GETS ENTERED INTO NSPIRES
% ================================

%%%%%%%%%%%%%%%%%%%%%%%%%%%%%%%%
%%%%%%%%%%%%%%%%%%%%%%%%%%%%%%%%
\section{Abstract}
\label{Sec:abstract}


%%%%%%%%%%%%%%%%%%%%%%%%%%%%%%%%
%%%%%%%%%%%%%%%%%%%%%%%%%%%%%%%%
\section{Introduction}
\label{Sec:introduction}

 \begin{figure*}[t!]
 \begin{center}
 % trim left bottom right top 
\includegraphics[trim={0 5 0 0}, clip, width=0.33\textwidth]{FIGURES/fig1.png}
\includegraphics[trim={0 0 0 0}, clip, width=0.66\textwidth]{FIGURES/fig2.png}
\vspace{-0.8cm}
\caption{Caption goes here}      
\label{fig:dataReductionFlowchart}
\end{center}
\end{figure*}

%%%%%%%%%%%%%%%%%%%%%%%%%%%%%%%%
%%%%%%%%%%%%%%%%%%%%%%%%%%%%%%%%
\section{Objectives}
\label{Sec:objectives}

%%%%%%%%%%%%%%%%%%%%%%%%%%%%%%
%%%%%%%%%%%%%%%%%%%%%%%%%%%%%%%%
\section{\MyName\ technical approach}
\label{Sec:methods}


%%%%%%%%%%%%%%%%%%%%%%%%%%%%%%%%
%%%%%%%%%%%%%%%%%%%%%%%%%%%%%%%%

\section{Scientific Goals and Perceived Impact of Proposed Work}
\label{Sec:Summary}

\subsection{Relevance to the Astronomical Community}

\newpage
%%%%%%%%%%%%%%%%%%%%%%%%%%%%%%%%%%%%%%%%%%%%%%%%%%%%%%%%%%%%%%%%%%%%%%%%
%%%%                     R E F E R E N C E S                       %%%%%
%%%%%%%%%%%%%%%%%%%%%%%%%%%%%%%%%%%%%%%%%%%%%%%%%%%%%%%%%%%%%%%%%%%%%%%%
\addcontentsline{toc}{section}{\Large References}
% To use the [1],[2],[3] style, use the below "modifiedIEEE"
% (note: this style is required for dual-anonymous reviews)
%\bibliographystyle{CLASS_FILES/modifiedIEEE}
% To use the "Smith et al 2004" style, use the below aasjournal.
\bibliographystyle{CLASS_FILES/aasjournal}
% To use a more compact "Smith+2004", use the below:
%\bibliographystyle{CLASS_FILES/aas4Proposals}
% Use the Google Drive-based BibFile Manager to easily construct the .bib file used below, and to easily add references to that file as you see the need while writing the proposal. 
\bibliography{bibFileManager.bib}

\newpage

%%%%%%%%%%%%%%%%%%%%%%%%%%%%%%%%%%%%%%%%%%%%%%%%%%%%%%%%%%%%%%%%%%%%%%%%
%%%%            D A T A   M A N A G E M E N T   P L A N            %%%%%
%%%%%%%%%%%%%%%%%%%%%%%%%%%%%%%%%%%%%%%%%%%%%%%%%%%%%%%%%%%%%%%%%%%%%%%%
%            >>>>>>>>>>  COMPONENT #4 <<<<<<<<<<<<<<<<<
\addcontentsline{toc}{section}{\Large Data Management Plan}
\begin{center}\color{NavyBlue}\normalfont\Large\textsc{Data Management Plan}\end{center}
\nosection{dmp}
\vspace{-0.5cm}
%%%%%%%%%%%%%%%%%%%%%%%%%%%%%%%%
%%%%%%%%%%%%%%%%%%%%%%%%%%%%%%%%
\label{Sec:dataManagementPlan}
\subsection{Data Types, Volumes, and Formats}
\label{subsec:dataTypesVolumesFormats}

\textbf{Proposed deliverables include:}\\

\subsection{Data \& Software Distribution}
\label{subsec:dataAvailability}

\newpage
%%%%%%%%%%%%%%%%%%%%%%%%%%%%%%%%%%%%%%%%%%%%%%%%%%%%%%%%%%%%%%%%%%%%%%%%
%%%%            B I O G R A P H I C A L   S K E T C H E S          %%%%%
%%%%%%%%%%%%%%%%%%%%%%%%%%%%%%%%%%%%%%%%%%%%%%%%%%%%%%%%%%%%%%%%%%%%%%%%
%            >>>>>>>>>>  COMPONENT #5 <<<<<<<<<<<<<<<<<
\addcontentsline{toc}{section}{\Large Biographical Sketches}
\begin{center}\color{NavyBlue}\normalfont\Large\textsc{Biographical Sketches}\end{center}
\label{Sec:bioSketches}
% The bio files are now located in a central place so that they can be continuously updated and used for all proposals simply by linking to them.  Maintenance will be substantially easier by having to keep track of a single folder within Overleaf.
\input{TABLES_BIOS/BIO-person1}
\newpage
\input{TABLES_BIOS/BIO-person2}

%%%%%%%%%%%%%%%%%%%%%%%%%%%%%%%%%%%%%%%%%%%%%%%%%%%%%%%%%%%%%%%%%%%%%%%%
%%%%          P E R S O N N E L   &   W O R K   E F F O R T        %%%%%
%%%%%%%%%%%%%%%%%%%%%%%%%%%%%%%%%%%%%%%%%%%%%%%%%%%%%%%%%%%%%%%%%%%%%%%%
%            >>>>>>>>>>  COMPONENT #6 <<<<<<<<<<<<<<<<<
\addcontentsline{toc}{section}{\Large Personnel and Work Effort}
\begin{center}\color{NavyBlue}\normalfont\Large\textsc{Personnel and Work Effort}\end{center}
\nosection{personnel}
\label{Sec:personnelWorkEffort}

\subsection{Personnel Overview}
\label{subsec:personnelOverview}

\subsection{Publication and Dissemination}

\subsection{Work Plan}
\label{subsec:timeline}
% Use the Google Drive-based Proposal Work PLan Tool to easily construct the NOTANONpersonnel_work_effort.tex file
\vspace{0.2cm}

The work effort (FTE and WYE) are shown in the following table:\\
\input{TABLES_BIOS/NOTANONpersonnel_work_effort}
\subsection{Project Milestones and Management}
% Use the the Google Drive-based Proposal Work Plan Tool to easily make the anonTimeline file
\input{TABLES_BIOS/anonTimeline}


\newpage
%%%%%%%%%%%%%%%%%%%%%%%%%%%%%%%%%%%%%%%%%%%%%%%%%%%%%%%%%%%%%%%%%%%%%%%%
%%%%         C U R R E N T  &  P E N D I N G  S U P P O R T        %%%%%
%%%%%%%%%%%%%%%%%%%%%%%%%%%%%%%%%%%%%%%%%%%%%%%%%%%%%%%%%%%%%%%%%%%%%%%%
%            >>>>>>>>>>  COMPONENT #7 <<<<<<<<<<<<<<<<<
\addcontentsline{toc}{section}{\Large Current and Pending Support}
\begin{center}\color{NavyBlue}\normalfont\Large\textsc{Current and Pending Support}\end{center}
\label{Sec:currentPendingSupport}
% update the pending and current support files in the separate Overleaf project  entited "BIO SKETCHES" 
\input{TABLES_BIOS/CurrentPendingSupport-person1}
\input{TABLES_BIOS/CurrentPendingSupport-person1}
\vspace{-0.5cm}
\newpage

%%%%%%%%%%%%%%%%%%%%%%%%%%%%%%%%%%%%%%%%%%%%%%%%%%%%%%%%%%%%%%%%%%%%%%%%
%%%%         S T A T E M E N T S  O F  C O M M I T M E N T         %%%%%
%%%%%%%%%%%%%%%%%%%%%%%%%%%%%%%%%%%%%%%%%%%%%%%%%%%%%%%%%%%%%%%%%%%%%%%%
%            >>>>>>>>>>  COMPONENT #8 <<<<<<<<<<<<<<<<<
\addcontentsline{toc}{section}{\Large Statements of Commitment}
\begin{center}\color{NavyBlue}\normalfont\Large\textsc{Statements of Commitment}\end{center}
\label{Sec:statementsOfCommitment}
All proposing team members have acknowledged their participation via NSPIRES.


%%%%%%%%%%%%%%%%%%%%%%%%%%%%%%%%%%%%%%%%%%%%%%%%%%%%%%%%%%%%%%%%%%%%%%%%
%%%%                        B U D G E T                           %%%%%
%%%%%%%%%%%%%%%%%%%%%%%%%%%%%%%%%%%%%%%%%%%%%%%%%%%%%%%%%%%%%%%%%%%%%%%%
%            >>>>>>>>>>  COMPONENT #9 <<<<<<<<<<<<<<<<<
\addcontentsline{toc}{section}{\Large Budget}
\begin{center}\color{NavyBlue}\normalfont\Large\textsc{Budget}\end{center}
\nosection{budget}
\label{sec:Budget}
\vspace{-0.5cm}
\subsection{Budget Narrative}
\label{subsec:budgetNarrative}

\subsection{Facilities and Equipment}


\newpage
\subsection{Other Direct Costs}
\input{TABLES_BIOS/NOTANONtravel_details}

\newpage
\input{TABLES_BIOS/NOTANONequipment_supplies_rolledUpTravel_publications}

\newpage
\subsection{Full Redacted Budget}

\end{document}
